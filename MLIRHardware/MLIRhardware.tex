\documentclass{tufte-handout}

%\geometry{showframe}% for debugging purposes -- displays the margins

\usepackage{amsmath}

% Set up the images/graphics package
\usepackage{graphicx}
\setkeys{Gin}{width=\linewidth,totalheight=\textheight,keepaspectratio}
\graphicspath{{graphics/}}

\title{MLIR for Hardware and Heterogeneous Accelerators}
\author{Stephen Neuendorffer and Samuel Bayliss, Xilinx Inc.}
%\date{24 January 2009}  % if the \date{} command is left out, the current date will be used

% The following package makes prettier tables.  We're all about the bling!
\usepackage{booktabs}

% The units package provides nice, non-stacked fractions and better spacing
% for units.
\usepackage{units}

% The fancyvrb package lets us customize the formatting of verbatim
% environments.  We use a slightly smaller font.
\usepackage{fancyvrb}
\fvset{fontsize=\normalsize}

% Small sections of multiple columns
\usepackage{multicol}

% Provides paragraphs of dummy text
\usepackage{lipsum}

% These commands are used to pretty-print LaTeX commands
\newcommand{\doccmd}[1]{\texttt{\textbackslash#1}}% command name -- adds backslash automatically
\newcommand{\docopt}[1]{\ensuremath{\langle}\textrm{\textit{#1}}\ensuremath{\rangle}}% optional command argument
\newcommand{\docarg}[1]{\textrm{\textit{#1}}}% (required) command argument
\newenvironment{docspec}{\begin{quote}\noindent}{\end{quote}}% command specification environment
\newcommand{\docenv}[1]{\textsf{#1}}% environment name
\newcommand{\docpkg}[1]{\texttt{#1}}% package name
\newcommand{\doccls}[1]{\texttt{#1}}% document class name
\newcommand{\docclsopt}[1]{\texttt{#1}}% document class option name

\begin{document}

\maketitle% this prints the handout title, author, and date

\begin{abstract}
\noindent MLIR\cite{lattner2020mlir} is a rapidly maturing compiler infrastructure focused on building and transforming abstractions, making it easier to build programming tools for new accelerators.  We believe that MLIR can also address tooling and design challenges in building those accelerators themselves.  This document outlines our vision in this area, with the goal to foster collaboration and shared concepts.\end{abstract}

%\printclassoptions


With the slowing rate of scaling in semiconductor process technology, there has become a widespread shift to develop more specialized architectures.  Circuits implementing these specialized architectures are generally called \emph{accelerators}, although they are developed as a tradeoff between performance, power usage, design cost and manufacturing cost. Accelerators typically implement compute structures, interconnect, and memory hierarchies optimized for particular workloads.  Optimizing memory and communication is often more important for overall design goals than optimizing the computing structures\cite{dallymemorywall}. Accelerators are almost always programmable to some extent, in order to allow them to operate on a set of similar workloads, although they are typically much less programmable than general purpose architectures.\begin{marginfigure} Tobias: Why accelerators only? What about the software stack for ASIC design. Is this not the very same stack? \end{marginfigure}\begin{marginfigure} Figure with gpu/fpga/asic tradeoff\end{marginfigure} Accelerators are often so valuable because of their specialization that they may be included in devices even when they are not constantly used\cite{hadi11darksilicon}.

The need for specialization has led to a wide variety of accelerators in some areas, such as machine learning\cite{hotchips2019}.  Programming a wide variety of accelerators from a wide scope of design languages is a significant challenge.  MLIR addresses this by using many abstractions called \emph{dialects}.  Individual design languages are typically associated with their own dialects, which are generally lowered into common dialects and optimized.  Implementing optimizations on common dialects allows the effort of developing and maintaining optimizations to be shared between design languages and target architectures.  Eventually, dialects are lowered (perhaps in a target-specific fashion) to low-level dialects and eventually to target-specific dialects, facilitating code generation for particular accelerator targets\cite{bondhugula2020high}.

At the same time, developing accelerators themselves is becoming more challenging\cite{berkeley06dwarfs}.  Accelerators are often complex and may have multiple design constraints which are difficult to meet.  Even building one accelerator can have a significant cost.  At the same time, accelerators are also numerous, implying that there is less design effort available for each accelerator and yet the cost of building an accelerator must be incurred over and over again.  We argue that the fundamental concepts of MLIR can address this hardware design complexity are as applicable to the process of \emph{building} accelerators as they are to the process of \emph{programming} them.  Furthermore, unifying the abstractions for accelerator design and accelerator programming is a key step towards enabling programming for arbitrary accelerators. 

\section{Dialects for Hardware Design}\label{sec:dialects}

In existing systems, we see the use of a wide variety of abstractions commonly in use. These abstractions tend to exist in a hierarchy of abstractions used at different parts of the design process, where higher-level abstractions are used to generate lower-level abstractions.  Often this hierarchy is not explicit in tools, but is only apparent in the informal processes applied by human engineers.  For instance, engineers might discuss the flow of data through a sequence of accelerators using a block diagram on a whiteboard, capture that architecture in informal documentation and then proceed with implementing and integrating these components in a hardware design language.  We believe that using MLIR to define hardware design abstractions using dialects will make these specifications more precise and enable more automation of transformations between these abstractions.

Note that these abstractions may be used in different ways at different points in the design process. These differences in usage often come down to a difference in component granularity.  For instance early in the design process an abstraction may be used with large, coarse-grained components. After lowering, perhaps through other abstractions, the same abstraction may again appear with smaller, find-grained components.\cite{ptolemy14SystemDesign}

Below, we summarize the most important abstractions used in existing processes.  This list is unlikely to be comprehensive and it is likely that the process of capturing these abstractions as MLIR dialects will identify additional abstractions or constraints that help facilitate the lowering between dialects.  Many existing systems span multiple abstractions, but we've tried here to focus on the most significant internal representations used in each tool.

\subsection{Structural Circuit Netlists}\label{sec:netlists}
Netlists are a longstanding abstraction used for circuit design, expressing the instantiation and interconnection of primitive components.   Components are typically fine-grained, representing logic gates in a standard-cell integrated circuit, or architectural primitives in programmable logic.  The EDIF\cite{EDIF} format has been almost ubiquitous since the early days of Electronic Design Automation, although the Verilog\cite{verilog} language is also commonly used to store netlists in modern tools.  

Netlist descriptions often blur the distinctions between the inputs and outputs of components.  Connections between components may represent physical connections where multiple components may (hopefully at different times!) drive a signal onto the same wire.  Alternatively, a single component may sometimes use a wire as an input and at other times use the same wire for output.

\subsection{Register-transfer level (RTL)}\label{sec:rtl}
The Register-transfer level is has been commonly used to describe logic circuits.  The process of logic synthesis is commonly used to lower RTL designs into structural netlists.  Verilog, SystemVerilog, and VHDL are commonly used in the industry.  The LLHD framework\cite{schuiki2020llhd} explicitly takes a multi-level approach to capturing RTL designs and lowering them into more fundamental constructs which can be efficiently simulated or synthesized.  A key aspect of this systems is the observation that useful input languages (such as SystemVerilog) provide a much wider set of constructs than can be easily synthesized, and managing this complexity in a structured way is key to building real systems.  Another recent system focused on improving RTL-level design is Chisel\cite{bachrach12chisel}, along with associated FIRRTL\cite{berkeley17firrtl}\cite{berkeley16firrtl} intermediate representation.

At the register-transfer level, the distinction between components which are clocked (i.e. registers) and those which are not clocked (i.e. combinational logic) is explicit.  Many circuit synthesis techniques have historically leveraged this distinction to focus only on the combinational aspects.  Modern circuit analysis techniques often consider the behavior across synchronous boundaries.  Retiming is a common transformation performed on RTL descriptions.

\subsection{Finite-State Machine + Datapath (FSMD)}\label{sec:fsmd}
High-Level synthesis(HLS) of RTL designs from C code has become a common paradigm for accelerator design\cite{cong11HLS}\cite{kastner2018parallel}.  Typically generating a high-performance implementation requires analysis and scheduling of the FSMD model in order to understand the performance implications of particular choices\cite{zhang13sdc}.

A common intermediate abstraction in HLS is a combination of finite-state machines and datapath abstractions.  Conceptually, in each state of the state machine, certain operations in a datapath are enabled.  This abstraction facilitates reasoning about resource sharing and pipelined behavior, while leaving the specifics of how those mechanisms are implemented implicit.  Lowering into an RTL-level description typically requires explictly representing these mechanisms, converting both the control finite-state machine and the datapath into registers and combinational logic. This lowering also requires explicitly representing clocks and clock enables in the circuit in an efficient way.  Note that finite-state machines may also be represented in other ways (for instance, using microcode) rather than being lowered through logic synthesis into fixed logic. 

Several projects have described the internal representation of HLS compilers.  LegUp\cite{canis13legup} leverages the Clang/LLVM framework for representing designs, while Bambu\cite{pilato13bambu}
builds an independent internal representation.  A recent effort in this area is Futil\cite{futil}, which implements an intermediate representation leveraged by the Dahlia system\cite{nigam20dahlia}.

\subsection{Dataflow/handshake}\label{sec:handshake}
Dataflow models have long been used to represent parallel computation.  Typically these models include components representing independent processes communicating through streams of data. These models provide a notion of concurrency which is intrinsically decoupled, unlike synchronous models like RTL.  This makes it easy to describe distributed systems with \emph{elastic} or \emph{latency-insensitive} properties, where the latency to communicate data from one process to another is guaranteed to not change the streams of data being computed.

It is possible to extract dataflow models from fine-grained CDFG representations of sequential code\cite{josipovic18hls} and the resulting dataflow models can be implemented directly in a circuit, although the circuits used to implement fine-grained handshake logic are often more complex than an equivalent FSMD model (for designs which can be statically scheduled).  More commonly, dataflow models are used with coarse-grained synchronous components to implement a globally asynchronous-locally synchronous(GALS) architecture.  This architecture is often appropriate for large accelerator designs where clock distribution and global timing closure can be challenging when implementing large componentized RTL designs.

\subsection{IP composition}\label{sec:ipi}
IP composition has become a key methodology for most FPGA and ASIC designs, enabling large designs to be constructed out of small components.  The key to making IP composition robust has been to focus on interface-based composition using protocols like ARM AXI, rather than signal-level composition which is commonly done in RTL design.  Standard interface protocols have enabled designers to independently solve the problems of how components behave internally with how they are integrated.  At the lowest levels of the design flow, this has allowed larger designs to meet timing during place and route, by enabling hierarchical floorplanning and late-stage interconnect synthesis.  It can also enable the impact of late-stage changes, called Engineering Change Orders (ECOs), to be encapsulated, avoiding unnecessary redundant verification.  IP composition also enables portable operating system software and runtimes based on declarative descriptions of device architectures, for instance using Device Trees\cite{deviceTrees}.

IPXact\cite{IPXact} is an industry standard file format, although a lack of industry-standard IP definitions and heavy reliance in practice on vendor extensions has meant that tools using IPXact are largely not inter-operable.  In addition, being based on XML, IPXact is somewhat verbose and difficult for humans to interact with.  DUH\cite{DUH} is a recent open-source effort based on JSON formats also focusing on IP composition.

\section{Tool Flows}\label{sec:flows}
Although it is important to have appropriate abstractions to build with, those abstractions are simply a means to support transformations of designs.  Although many flows are potentially interested, we highlight some of the most interesting flows.

\subsection{High-Level Synthesis (HLS)}\label{sec:HLS}
A common toolflow is to synthesize circuits from sequential algorithmic code.  Many tools have implemented this type of tool flow, usually with a focus on signal processing applications.  In such a flow, building deeply pipelined and parallelized implementations of loops is a key capability.  A typical high-level synthesis flow starts from an optimized control-dataflow graph (CDFG).  Additional hardware-oriented optimizations, such as bitwidth minimization, ROM inference, shift-register identification, and algebraic optimizations with hardware cost models taken into account are typically performed to reduce implementation cost.  Memory analysis and optimization is also critical for most code.  Arrays are often implemented in local memories wherever possible, and memory accesses optimized to eliminate cache hierarchy wherever possible.  Memories are often restructured through memory partitioning to increase bandwidth.  Accurate points-to analysis and affine index analysis are necessary to statically identify limits on pipelining.   Operation scheduling is performed to identify initial clock-cycle boundaries.

A scheduled design can be expressed as an FSMD model.  In this form, it becomes possible to identify micro-architectural optimizations.  Operations can be shared on the same hardware and the multiplexing cost from this sharing can be accurately estimated.  Control logic cost can also be estimated and minimized.  Verification is also important in this stage: simulation can be used to verify a design with the original sequential testbench and partial formal analysis, e.g. reachability analysis on generated control logic, are possible.  Interface synthesis is also applicable at this level of abstraction, converting algorithmic interfaces (like sequential accesses to memory) with implementation interfaces (like FIFO interfaces).

After optimization, the FSMD model can be realized as a Register-Transfer Level (RTL) design.  This transformation makes the control FSM explicit and blurs the distinction between the control logic and data path.   Logic optimization is possible between the control logic and the data path and these optimizations can leverage Clock Enable inputs on registers.  RTL models are typically realized as Verilog or VHDL, enabling implementation using vendor tools.  Alternatively, these models could be further lowered to more hardware-specific dialects in MLIR.

\subsection{Dataflow-based Multicore Programming}\label{sec:dataflowmulticore}

Multicore programming is commonly done using Single-Program Multiple-Data (SPMD) programming, such as CUDA for GPUs.  Typically, each processor reads a portion of a large data set from external memory into local memory, processing it locally before writing results back to external memory.  As a result, communication between processors largely happens through external memory, resulting in significant power consumption and a communication bottleneck.  While communication between cores using on-chip memory is also possible, this must be managed explicitly and separately from global memory.  Explicit dataflow-based programming represents inter-process communication explictly as streams of data, allowing communication intent to be explicitly separated from stateful storage.  Additionally, since dataflow processes do not share state, dataflow-based programming describes memory locality explicitly in a program.

Many optimizations are possible on dataflow models.  Most critically, \em{fusion} optimizations manipulate the dataflow graph to generate toplevel dataflow components where each component represents execution on a single processor.\cite{gordon02streamit}  In order to generate a balanced system, this is typically done with several goals.  Primarily, this needs to be done considering locality of access, often merging dataflow components which share inputs or process data in a pipeline.  Secondarily, it is also important to balance the amount of processing done on each physical processor.  This minimizes the amount of time where one processor is stalled waiting for data from another processor.  In more complicated systems with feedback, unbalanced workloads may be preferable to ensure that feedback paths complete quickly.  \em{Fission} transformations may also be important, where large granularity components exist in a dataflow program.  Obviously data-parallel computations can be partitioned using Split/Join pairs, while more complex components consisting of nested loops with recurrences require more complex analysis.\cite{rijpkema00compaan} 

Stream computation can be implemented using different mechanisms depending on the architecture. Some architectures provide stream operations in the ISA, enabling data to be communicated on physical streams. However, in most cases, stream accesses are turned into memory operations.  For instance, streams on general purpose CPUs are typically implemented through memory allocation, pointer manipulation and synchronization between threads or perhaps using OS-level pipes between processes.  Stream communication may also occur between general purpose CPUs and an accelerator with stream communication being implemented by a combination of DMA hardware, shared memory access, and runtime API calls.  In distributed systems, stream communication can also be implemented over a network, for instance using TCP/IP.  In such cases, understanding the bottleneck of a shared medium can be critical.  In each of these cases, there are lowerings from dataflow concepts to more generic process/thread/fibre concepts, where there are significant architecture-specific tradeoffs which must be managed.  We argue that modeling this lowering explicitly is almost always more convenient for a programmer than exposing low-level concepts (e.g. threads and locks) to the programmer.



\subsection{}

\bibliography{MLIRhardware}
%\bibliographystyle{hplain}
\bibliographystyle{plain}

\end{document}
